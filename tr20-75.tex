\documentclass{article}
\begin{document}
\title{Why Quadratic Log-Log Dependence Is Ubiquitous And What Next}
\author{Sean R. Aguilar$^1$, Vladik Kreinovich$^1$, and Uyen Pham$^2$\\
$^1$Department of Computer Science\\ University of Texas at El
Paso\\500 W. University\\ El Paso, TX 79968,
USA\\sraguilar4@miners.utep.edu, vladik@utep.edu\\ $^2$University
of Economics and Law and\\John Von Neumann Institute\\Ho Chi Minh
City, Vietnam\\uyenph@uel.edu.vn}

\date{}
\maketitle

\begin{abstract}
In many real-life situations ranging from financial to volcanic data,
growth is described either by a power law -- which is linear in log-log scale, or by
a quadratic dependence in the log-log scale. In this paper, we use natural scale invariance
requirement to explain the ubiquity of such dependencies. We also explain what should be a
reasonable choice of the next model, if quadratic turns out to be not too accurate: it turns out that
under scale invariance, the next class of models are cubic dependencies in the log-log scale, then
fourth order dependencies, etc.
\end{abstract}

\section{Formulation of the Problem}

\noindent{\bf Predictions: a typical situation.} One of the main objective of science and
engineering is to predict the future state of the world -- i.e., the future values of the
quantities that describe this state -- and to come up with measures that lead to the most
favorable future state. For example, we want to predict tomorrow's weather -- and if it will
be catastrophic in a given area, we need to plan corresponding closings and, if needed, evacuations.
We want to predict next year's GDP -- and if the current trends predict a crisis, we want to come
up with measures that would prevent this crisis -- or at least decrease its severity.

In some situations,
we can predict the future values of some quantities with high accuracy.
For example, we can predict Solar eclipses centuries ahead.

However, such situations are rare. In most real-life situations, we cannot make exact
predictions: e.g., when we predict the weather, there are many factors that affect
tomorrow's weather and that we, at present, do not know. In such situations,
we can predict, at best, the probabilities of future values of the corresponding quantity.
These probabilities can be described, e.g., by the probability density function (pdf) $\rho(x)$.
Such situations are typical in economics and finance, they are also typical in geosciences -- e.g., in predicting
volcanic activity, they are typical in many other application areas.
\medskip

\noindent{\bf Stationary vs. non-stationary situations.} In some cases -- e.g., in celestial mechanics or,
for a reasonably short period of time, in weather prediction -- the corresponding probabilities remain
the same day after day and year after year. So, the probability density function $\rho_t(x)$ remains the same for all moments of time $t$:
$\rho_t(x)=\rho(x).$

However, in many other cases, the situation changes with time: on average, the values $x$ grow with time.
This happens with economic characteristics such as GDP or stock prices, this happens with
volcanic activity when the volcano becomes more and more active. In many such cases, the shape of the
probability distribution remains the same, but the scale changes. In other words, at each moment of time $t$,
the distribution of $x$ is similar to the initial ($t=1$) distribution of the quantity $\displaystyle\frac{x}{C(t)}$ for some
increasing function $C(t)$, i.e., $\rho_t(x)$ should be proportional to $\rho_1\left(\displaystyle\frac{x}{C(t)}\right)$. The coefficient of proportionality can be easily found from the condition that the overall probability should be equal to 1, i.e., that $\int \rho_t(x)\,dx=1$. Thus, we get
$$\rho_t(x)=\frac{1}{C(t)}\cdot \rho\left(\frac{x}{C(t)}\right).\eqno{(1)}$$
\medskip

\noindent{\bf How $C(t)$ depends on time: empirical fact.} In many practical situations, the growth $C(t)$
is described by the power law:
$$C(t)=A\cdot t^b,\eqno{(2)}$$ for some constants $A$ and $b$. If we take the logarithm of both sides of this formula, we can conclude that
in the log-log scale, the power law becomes a linear dependence:
$$\ln(C(t))=b\cdot \ln(t)+\ln(A).\eqno{(3)}$$
In other cases, we have a more complex dependence $$C(t)=A\cdot t^{b(t)}\eqno{(4)}$$ where $$b(t)=b_0+b_1\cdot \ln(t).\eqno{(5)}$$
In this case, in log-log-scale, we have
$$\ln(C(t))=b(t)\cdot \ln(t)+\ln(A)=b_0\cdot \ln(t)+b_1\cdot (\ln(t))^2+\ln(A).\eqno{(6)}$$
In other words, we have a quadratic log-log dependence, of which the linear log-log dependence (3) is a particular case
corresponding to $b_1=0$.

Such dependencies is really ubiquitous: e.g., an empirical analysis provided in \cite{Mariani 2020} have shown that
many real-life dependencies, ranging from economic to volcanic data, follow these formulas. This naturally
leads to the following questions.
\medskip
\newpage

\noindent{\bf Natural questions.}
\begin{itemize}
\item How can we explain the ubiquity of the dependence (4)-(5)?
\item When the match with this formula is not perfect, what more accurate formula should we try?
\end{itemize}
In this paper, we provide answers to both questions.

\section{Analysis of the Problem: Why Power Law?}

\noindent{\bf Why power law.} In many case, the dependence of $C(t)$ on $t$ is described by the power law.
So, before we analyze the more general question of why the general dependence (4)-(5) is ubiquitous,
let us analyze why power law is frequently observed.
\medskip

\noindent{\bf Scale-invariance: idea.} We are interested in learning how the growth function $C(t)$ depends on time $t$.
In our analysis, we use numerical values of $C(t)$ and numerical values of time $t$. Numerical values of time
depend on the measuring unit: we get different values if we use years, quarters, months, days, etc. If we replace
the original unit with the one which is $\lambda$ times smaller, then all numerical value of time $t$ are replaced
with new values $\lambda\cdot t$. For example, if we replace years with months, then in the new units, the period of 2 years
becomes $12\cdot 2=24$ months.

In many cases, we do not have a preferred unit for measuring time. In such situations, it is reasonable to
require that the dependence between $C$ and $t$ remains the same if we change the unit for measuring time.
This requirement is known as {\it scale invariance}.

The requirement of scale-invariance is typical in physics: e.g., the formula $d=v\cdot t$ describing the relation between
distance $d$, velocity $v$, and time $t$ remains true no matter what units we select for measuring time, e.g., hours or seconds.

Of course, we cannot simply assume that the value $C(t)$ remains the same: the growth in two years is clearly different from the
growth in two months. This can be easily explained on the example of the above physics formula: if we change the unit for time, then, for the formula to remain valid, we need to also appropriately change the units for other quantities: e.g., replace the velocity unit from km/hour to km/sec. In our cases, this means that if we change a unit for time, then the formula $C=C(t)$ should remain valid if we also appropriately change the unit for $C$, into a new unit which is $\mu$ times smaller, for some $\mu$ depending on $\lambda$.

In other words, if we have $$C=C(t),\eqno{(7)}$$ then, for every $\lambda$, we should also have $$C'=C(t'),\eqno{(8)}$$
where $$C'=\mu(\lambda)\cdot C\mbox{  and  }t'=\lambda\cdot t.\eqno{(9)}$$
\medskip

\noindent{\bf Scale-invariance explains power law.}
Substituting expressions (9) for $C'$ and $t'$ into the formula (8), we conclude that $\mu(\lambda)\cdot C=C(\lambda\cdot t).$ Substituting the expression (7) for $C$ into this formula, we conclude that
$$C(\lambda\cdot t)=\mu(\lambda)\cdot C(t).\eqno{(10)}$$
In real-life, the dependencies are usually continuous, and for continuous functions, it is known that all solutions of the function equation (10) have the form $C(t)=A\cdot t^b$; see, e.g., \cite{Aczel 2008}.

Thus, we indeed have a natural explanation for the power law.

\section{How to Explain Quadratic Log-Log Dependence}

\noindent{\bf Main idea.} As we have mentioned earlier, the main reason why the real-life processes are probabilistic is that
the actual value $x(t)$ depends also on some characteristics that we do not know. In particular, this means that the value $C(t)$ also depends on such characteristics $y_1,\ldots,y_n$: $C=C(t,y_1,\ldots,y_n)$.

Let us first consider the simplest such case, when we consider only the dependence of one such characteristics $y_1$, then $C=C(t,y_1)$.
\medskip

\noindent{\bf Let us apply scale-invariance to this situation.} Now, in addition to time $t$, we have another quantity $y_1$ for which we can also
select different measuring units. It is therefore reasonable to require that no matter how we change both unit for $t$ and unit for $y_1$,
we will get the same dependence.

In precise terms, for every $\lambda>0$ and $\lambda_1>0$, there exists a value $\mu(\lambda,\lambda_1)$ such that if $$C=C(t,y_1),\eqno{(11)}$$ then
$$C'=C(t',y'_1),\eqno{(12)}$$ where $$t'=\lambda\cdot t,\ \  y'_1=\lambda\cdot y_1,\ \  C'=\mu(\lambda,\lambda_1)\cdot C.\eqno{(13)}$$
\medskip

\noindent{\bf Let us analyze this situation.} Substituting the expressions (13) into the formula (12), we conclude that
$$C(\lambda\cdot t,\lambda_1\cdot y_1)=\mu(\lambda,\lambda_1)\cdot C.\eqno{(14)}$$ Substituting the expression (11) for $C$ into the formula (14), we get
$$C(\lambda\cdot t,\lambda_1\cdot y_1)=\mu(\lambda,\lambda_1)\cdot C(t,y_1).\eqno{(15)}$$
For each $y_1$, by taking $\lambda_1=1$, we conclude that
$$C(\lambda\cdot t,y_1)=\mu(\lambda,1)\cdot C(t,y_1).\eqno{(16)}$$
Thus, for each $y_1$, the function $C_{y_1}(t)\stackrel{\rm def}{=}C(t,y_1)$ satisfies the formula (10). Thus, based on the result cited in the previous section, we have
$$C_{y_1}(t)=C(t,y_1)=A(y_1)\cdot t^{b(y_1)},\eqno{(17)}$$ for some $A$ and $b$ depending on $y_1$. In particular, in log-log scale, we get
$$\ln(C(t,y_1))=b(y_1)\cdot \ln(t)+\ln(A(y_1)).\eqno{(18)}$$

Similarly, for every $t$, we can take $\lambda=1$ and get
$$C(t,\lambda_1\cdot y)=\mu(1,\lambda_1)\cdot C(t,y_1).\eqno{(19)}$$
Thus, for each $t$, the function $C_{t}(y_1)\stackrel{\rm def}{=}C(t,y_1)$ satisfies the formula (10). Thus, based on the result cited in the previous section, we have
$$C_{t}(y_1)=C(t,y_1)=A'(t)\cdot y_1^{b'(t)},\eqno{(20)}$$ for some $A'$ and $b'$ depending on $t$. In particular, in log-log scale, we get
$$\ln(C(t,y_1))=b'(t)\cdot \ln(y_1)+\ln(A'(t)).\eqno{(21)}$$

The formulas (18) and (21) describe the same expression $\ln(C(t,y_1))$. By equating these expressions for two different
values $t_1<t_2$, we conclude that
$$b(y_1)\cdot \ln(t_1)+\ln(A(y_1))=b'(t_1)\cdot \ln(y_1)+\ln(A'(t_1));\eqno{(22)}$$
$$b(y_1)\cdot \ln(t_2)+\ln(A(y_1))=b'(t_2)\cdot \ln(y_1)+\ln(A'(t_2)).\eqno{(23)}$$
Subtracting (22) from (23), we get
$$b(y_1)\cdot (\ln(t_2)-\ln(t_1))=(b'(t_2)-b'(t_1))\cdot \ln(y_1)+(\ln(A'(t_2))-\ln(A'(t_1)).\eqno{(24)}$$
So, by dividing both sides by the difference $\ln(t_2)-\ln(t_1)$, we get
$$b(y_1)=c_1\cdot \ln(y_1)+c_2,\eqno{(25)}$$
where we denoted $$c_1=\frac{b'(t_2)-b'(t_1)}{\ln(t_2)-\ln(t_1)}\mbox{  and  }
c_2=\frac{\ln(A'(t_2))-\ln(A'(t_1))}{\ln(t_2)-\ln(t_1)}.\eqno{(26)}$$

Similarly, if we multiply (23) by $\ln(t_1)$, (22) by $\ln(t_2)$, and subtract the results, we get
$$\ln(A(y_1))\cdot (\ln(t_2)-\ln(t_1))=(b'(t_2)\cdot \ln(t_1)-b'(t_1)\cdot \ln(t_2))\cdot \ln(y_1)+$$
$$(\ln(A'(t_2))\cdot \ln(t_1)-\ln(A'(t_1))\cdot \ln(t_2)),\eqno{(27)}$$
hence
$$\ln(A(y_1))=c_3\cdot \ln(y_1)+c_4,\eqno{(28)}$$
where we denoted
$$c_3=\frac{b'(t_2)\cdot \ln(t_1)-b'(t_1)\cdot \ln(t_2)}{\ln(t_2)-\ln(t_1)}\eqno{(29{\rm a})}$$
and
$$c_4=\frac{\ln(A'(t_2))\cdot \ln(t_1)-\ln(A'(t_1))\cdot \ln(t_2)}{\ln(t_2)-\ln(t_1)}.\eqno{(29{\rm b})}$$

Substituting the expressions (25) and (28) into the formula (18), we conclude that
$$\ln(C(t,y_1))=(c_1\cdot \ln(y_1)+c_2)\cdot \ln(t)+(c_3\cdot \ln(y_1)+c_4)=$$
$$c_4+c_2\cdot \ln(t)+c_3\cdot \ln(y_1)+c_1\cdot \ln(t)\cdot \ln(y_1).\eqno{(30)}$$

If we now assume that the dependence of $y_1$ on $t$ is also scale-invariant, then the result
of the previous section shows that
$$y_1=A''\cdot t^{b''}$$ for some $A''$ and $b''$, i.e., in log-log form,
$$\ln(y_1)=b''\cdot \ln(t)+\ln(A'').\eqno{(31)}$$
Substituting the expression (31) into the formula (30), we get
$$\ln(C(t))=\ln(C(t,y_1(t)))=C_0+C_1\cdot \ln(t)+C_2\cdot (\ln(t))^2,\eqno{(32)}$$
where $$C_0=c_4+c_3\cdot \ln(A''),\ \  C_1=c_2+c_3\cdot b''+c_1\cdot \ln(A''),\ \
C_2=c_1\cdot b''.\eqno{(33)}$$
Thus, we indeed explained the quadratic log-log dependence!

\section{What Next?}

\noindent{\bf Let us use scale-invariance.}
In general, we have a dependence
$$C=C(t,y_1,\ldots,y_n),\eqno{(34)}$$ on $n\ge 1$ auxiliary quantities. In this general case, scale-invariance means
for every $\lambda>0$ and for all possible values $\lambda_1>0$, \ldots, $\lambda_n$, there exists a value $\mu(\lambda,\lambda_1,\ldots,\lambda_n)$ such that if (34) is satisfied, then
$$C'=C(t',y'_1,\ldots,y'_n),\eqno{(35)}$$ where $$t'=\lambda\cdot t,\ \  y'_1=\lambda\cdot y_1,\ldots,
y'_n=\lambda\cdot y_n,\ \ C'=\mu(\lambda,\lambda_1,\ldots,\lambda_n)\cdot C.\eqno{(36)}$$
\medskip

\noindent{\bf What we can derive from scale-invariance.}
Similar to the previous section, we can thus conclude that the expression $\ln(C(t,y_1,\ldots,y_n))$ is linear in $\ln(t)$, linear in $\ln(y_1)$, \ldots, and
linear in $\ln(y_n)$. Thus, it is a multi-linear function:
$$\ln(C(t,y_1,\ldots,y_n))=c_0+c_t\cdot \ln(t)+\sum_{i=1}^n c_i\cdot \ln(y_i)+$$
$$\sum_{i=1}^n c_{t,i}\cdot \ln(t)\cdot \ln(y_i)+\sum_{i<j} c_{i,j}\cdot \ln(y_i)\cdot \ln(y_j)+\ldots+$$ $$c_{t,1,\ldots,n}\cdot \ln(t)\cdot \ln(y_1)\cdot \ldots\cdot \ln(y_n).\eqno{(37)}$$

If we assume that the dependence of each auxiliary quantity $y_i$ on $t$ is also scale-invariant, then we get
$$\ln(y_i)=b''_i\cdot \ln(t)+\ln(A''_i)\eqno{(38)}$$ for some values $b''_i$ and $A''_i$. Substituting the expressions (38) into the formula
(37), we conclude that
$$\ln(C(t))=C_0+C_1\cdot \ln(t)+C_2\cdot (\ln(t))^2+\ldots+C_{n+1}\cdot (\ln(t))^{n+1}.\eqno{(39)}$$
\medskip

\noindent{\bf Resulting recommendation.} So, if quadratic log-log dependence (corresponding to $n=1$) is too inaccurate, we need to try cubic log-log dependence (corresponding to $n=2$), then, if needed, fourth order log-log dependence corresponding to $n=3$, etc.

\section*{Acknowledgments}

This work was supported in part by the National Science Foundation
grants 1623190 (A Model of Change for Preparing a New Generation
for Professional Practice in Computer Science) and HRD-1242122
(Cyber-ShARE Center of Excellence).

\begin{thebibliography}{9}

\bibitem{Aczel 2008}
J. Acz\'el and J. Dhombres, {\it Functional Equations in Several
Variables}, Cambridge University Press, 2008.

\bibitem{Mariani 2020} M. C. Mariani, P. K. Asante, M. A. M. Bhuyian, M. P. Beccar-Varela, S.~Jaroszewicz, and O. K. Tweneboah,
``Long-range correlations and characterization of financial and volcanic time series'', {\it Mathematics}, 2020, Vol.~2020,
No. 8, Paper 441.

\end{thebibliography}
\end{document}
